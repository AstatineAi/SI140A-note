% 定义
% \begin{definition}{标题}{引用标签}\index{glossary用的拼音@glossary 内容}
% content
% \end{definition}

% 定理
% \begin{theorem}{标题}{引用标签} 
% content
% \end{theorem}

% \begin{equation}\label{eq:2:引用标签}
%   content
% \end{equation}

% 引用定义
% \autoref{def:引用标签}

% 引用定理
% \autoref{thm:引用标签}

% 引用等式
% \autoref{eq:2:引用标签}

% 同理,引理 -> lemma lem
%       例子 -> example ex
%       推论 -> corollary cor

\chapter{概率与计数 Probability and Counting}

\section{概率模型 Probabilistic Model}

\begin{definition}{集合}{集合}\index{jihe@集合 (Set)}
    
    \begin{enumerate}
        \item 空集: $\emptyset$
        \item 子集(subset)关系:$A \subseteq B$
        \item 交(union):$A \cup B$
        \item 并(intersection):$A \cap B$
        \item 补(complement):$A^c$
        \item De Morgan 律:$(A \cup B)^c = A^c \cap B^c, (A \cap B)^c = A^c \cup B^c$
    \end{enumerate}
    
\end{definition}

\begin{definition}{样本空间(Sample Space)}{样本空间}
    一个实验的全部可能的结果的集合称为\term{样本空间}.
\end{definition}

\begin{definition}{事件(event)}{事件}
    样本空间的子集称为\term{事件}.
    
    当实际结果属于一个事件时,称这个事件发生 (occured)
\end{definition}


\begin{tabular}{ll}
    \toprule
    \textbf{English} & \textbf{Sets} \\
    \midrule
    \textit{Events and occurrences} & \\
    sample space & \(S\) \\
    \(s\) is a possible outcome & \(s \in S\) \\
    \(A\) is an event & \(A \subseteq S\) \\
    \(A\) occurred & \(s_{\text{actual}} \in A\) \\
    something must happen & \(s_{\text{actual}} \in S\) \\
    \midrule
    \textit{New events from old events} & \\
    \(A\) or \(B\) (inclusive) & \(A \cup B\) \\
    \(A\) and \(B\) & \(A \cap B\) \\
    not \(A\) & \(A^c\) \\
    \(A\) or \(B\), but not both & \((A \cap B^c) \cup (A^c \cap B)\) \\
    at least one of \(A_1, \ldots, A_n\) & \(A_1 \cup \cdots \cup A_n\) \\
    all of \(A_1, \ldots, A_n\) & \(A_1 \cap \cdots \cap A_n\) \\
    \midrule
    \textit{Relationships between events} & \\
    \(A\) implies \(B\) & \(A \subseteq B\) \\
    \(A\) and \(B\) are mutually exclusive & \(A \cap B = \emptyset\) \\
    \(A_1, \ldots, A_n\) are a partition of \(S\) & \(A_1 \cup \cdots \cup A_n = S, \, A_i \cap A_j = \emptyset \text{ for } i \neq j\) \\
    \bottomrule
\end{tabular}

\newpage

\section{概率与计数的朴素定义}

\subsection{概率}

\begin{definition}{概率(朴素定义)}{概率naive}
    假设:
    \begin{enumerate}
        \item 有限样本空间
        \item 输出结果等可能
    \end{enumerate}
    
    则:
    
    令 $A$ 为一个事件,$S$ 为其样本空间,则 $A$ 的概率为
    
    $$
    P_{naive}(A) = \frac{|A|}{|P|} = \frac{\text{number of outcomes favorable to } A}{\text{total number of outcomes in } S}
    $$
    
    在这种情况下概率被转化为计数问题。
\end{definition}

\begin{example}{Pascal-Fermat Correspondence}{PF Bet Game}
    Alice 和 Bob 抛硬币,三次获胜的人赢。每一轮 Alice 赌注硬币为正面 (Head),Bob 赌注反面 (Tail)。当前比分为 $2 : 1$,若此时结束游戏该如何按可能的结果划分奖品?\\
    
    在第五轮游戏必然结束,则剩下两局样本空间为:$S = \{HH, HT, TH, TT\}$,当且仅当投出 $TT$ 时 Bob 获胜,则应该给 Alice $\dfrac{3}{4}$,给 Bob $\dfrac{1}{4}$
\end{example}

\subsection{计数}

\begin{itemize}
    \item \textbf{Sampling}: 从集合等可能获取一个元素
    \item \textbf{With Replacement \& without replacement}: 取出后放回/不放回
    \item \textbf{Ordered \& Unorderd}
\end{itemize}

\begin{example}{生日问题}{生日问题}
    房间内有 $k$ 人,假设一个人生日为一年 $365$ 天内等可能随机一天,生日互相独立,求至少两人有相同生日的概率。\\
    
    相当于从集合 $\{1, 2, 3, \dots, 365\}$ 取出 $k$ 个元素,with replacement。令 $A$ 表示“存在两人以上……“的情况,则 $A^c$ 为”没有任何人和别人生日相同“的情况。\\
    
    $$
    P(A^c) = \frac{|A^c|}{|S|} = \frac{\text{without replacement 的 samples}}{\text{with replacement}} = \frac{365!}{365^k} (k \leqslant 365)
    $$
    
    $$
    P(A) = 1 - P(A^c)
    $$
    
\end{example}

\begin{theorem}{广义生日问题}{GBirthday}
    从 $n$ 个值中随机选 $k$ 次,当 $k \approx 1.18 \sqrt{n}$ 时,有 50\% 的概率,至少有两个选取的值相同。\\
    
    考虑按每个随机变量来计算 $P(A^c)$, 则第一个变量从 $[1, n]$ 选择任意,第二变量必须取到和第一个不同的,第三个取和一、二不同的,并且变量之间相互独立,以此类推:
    
    $$
    P(A^c) = \frac{n}{n} \times \frac{n - 1}{n} \times \frac{n - 2}{n} \cdots \times \frac{n - (k - 1)}{n} = \prod_{i = 1}^{k} \frac{n - i}{n}
    $$
    
    当 $n \gg k$ 时,$\dfrac{n - i}{n} = 1 - \dfrac{i}{n} \approx \mathrm{e}^{-\frac{i}{n}}$,于是 $\prod \frac{n - i}{n} = \mathrm{e}^{-\frac{1 + 2 + \cdots + k - 1}{n}} = \mathrm{e}^{-\frac{k (k - 1)}{2n}} \approx \mathrm{e}^{-\frac{k^2}{2n}}$
    
    于是 $P(A) \approx 1 - \mathrm{e}^{-\frac{k^2}{2n}}$
    
    代入 $P(A) = 0.5$ 得出 $k = \sqrt{n \cdot 2 \ln 2} \approx 1.18 \sqrt{n}$
    
\end{theorem}

\begin{theorem}{多项式定理(Multinomial Theorem)}{Multinomial Theorem}
    $$
    (x_1 + x_2 + x_3 + \dots + x_r)^n = \sum_{n_1, n_2, \dots, n_r \geqslant 0} \frac{n!}{n_1 ! n_2 ! \cdots n_r !} x_1^{n_1} x_2^{n_2} \cdots x_r^{n_r} (n_1 + n_2 + \cdots + n_r = n)
    $$
    
    组合意义:把 $n$ 个不同的人分成 $r$ 组
\end{theorem}

在证明组合恒等式时,可以使用式子两侧组合意义说明等式成立。

\begin{theorem}{吸收/提取恒等式}{AEIdentity}
    $$
    n \binom{n - 1}{k - 1} = k \binom{n}{k}
    $$
\end{theorem}

\begin{proof}
    假设有如下场景:从 $n$ 人中选出 $k$ 个组成小队,并且这 $k$ 人中有一人为小队长。\\

    \autoref{thm:AEIdentity} LHS 可以视为先选出小队长($n$ 种方案),然后在剩下的 $n - 1$ 人中选出小队剩余成员(组合数). RHS 可以视为先从 $n$ 人中选出组成小队的 $k$ 人(组合数),然后在小队内部选出队长($k$ 种方案),可知等式成立。
\end{proof}

\begin{theorem}{Vandermonde 卷积}{Vandermonde}
    
    $$
    \binom{m + n}{k} = \sum_{j = 0}^{k} \binom{m}{j} \binom{n}{k - j}
    $$
    
\end{theorem}

\begin{proof}
    
\end{proof}

选择 \( n \) 个对象中的 \( k \) 个对象,可能的方式数量:

\begin{center}

\begin{tabular}{|c|c|c|}
\hline
& 顺序有关 & 顺序无关 \\
\hline
允许放回 & \( n^k \) & \( \displaystyle \binom{n+k-1}{k} \) \\
\hline
不允许放回 & \( n(n-1) \cdots (n-k+1) \) & \( \displaystyle \binom{n}{k} \) \\
\hline
\end{tabular}

\end{center}
